\documentclass{article}

\usepackage{bm}
\usepackage{amsmath}
\usepackage{amssymb}
\usepackage[citestyle=authoryear]{biblatex}
\addbibresource{final.bib}


\author{Tom Wallace}

\title{STAT 778 Final Project: Igeta, Takahashi, and Matsui
2018}

\begin{document}

\maketitle

\section{Introduction}

\subsection{Overview}

This paper documents a simulation study based on \cite{igeta2018}. It
gives some background on their method; presents a software program coded in C; 
and conducts a simulation study using this program. 

\subsection{Background}

Overdispersion refers to a situation in which the variance of a dataset exceeds
that expected under the assumed statistical distribution. 
Overdispersion is common in count data. As a motivating example, consider the
Poisson distribution. A single parameter determines both the 
mean and variance: i.e., for $X \sim \mathrm{Poisson}(\lambda)$, $\mu =
\sigma^2 = \lambda$. As a consequence, if the mean and variance
differ in a dataset, the researcher cannot adjust parametric assumptions for 
without also affecting the other. If not addressed, 
overdispersion results in distorted test statistics and estimated
standard errors. Clinical trials often feature count data: e.g., a trial of an
anti-epsilepsy treatment may use \textit{number of seizures over study period} as the outcome
variable. The serious consequences of statistical error in such settings demands
a rigorous method for dealing with overdispersion. \cite{igeta2018} is a new
entry to the large literature on this topic. In particular, it presents methods
for calculating statistical power and sample size in the presence of
misspecified variance.

\section{Methods}

The following is a brief summary of the methods proposed in \cite{igeta2018}.
The reader is encouraged to consult their paper for the necessary details. All
notation here follows that used in the original paper.

Consider a randomized control trial featuring $n$ subjects. 
Subjects are randomly assigned to a treatment or control group.
Let $n_i$ be the sample size of the $i$th group, with $i \in \{1, 2\}$.
Let $X_{ij}$ be an indicator variable of subject $j$'s group assignment. 
$X_{ij}=0$ indicates assignment to the control group.
Let $Y_ij$ be a count variable for patient $j$ in group $i$ over the follow-up period $[0, T_{ij}]$.
The expected value of $Y_{ij}$ is affected by a rate parameter $\lambda_i$.
The goal is to estimate the effect of treatment via a Poisson model 
$$
\lambda_i 
= \exp{(\beta_0 + \beta_1 X_{ij})}
$$
but $\lambda$ is overdispersed. 

\cite{igeta2018} propose a procedure for determining sample size and power in
the presence of overdispersion. The basic idea is that we assume there is some 
true variance function $V$ but this function is unknown. In practice we use
working variance function $\tilde{V}$. We don't actually know if we have
properly specified the true variance function and so would like sample size and power
calculations that are robust to misspecification.

\cite{igeta2018} employ a Wald-type test statistic using the sandwich-type
robust variance estimator under the null hypothesis:

\begin{equation}
	Z = \frac{\hat{\beta}_1}{\sqrt{n^{-1} \hat{W}_0}}
\end{equation}

They propose that the asymptotic power of the test using $Z$ 
with two-sided significance level $\alpha$ is:

\begin{equation}
	\mathrm{Pr} \left( Z > z_{1 - \alpha/ 2} \right) 
	=
	1 - \Phi 
	\left(
	z_{1 - \alpha / 2}
	\sqrt{\frac{W_0}{W_1}}
	-
	\sqrt{n}
	\frac{\beta_1}{\sqrt{W_1}}
	\right)
\end{equation}

The sample size that provides power greater than or equal to $1 - \beta$ is

\begin{equation}
	n \geq \frac{(z_{1 - \alpha / 2} \sqrt{W_0} + z_{1 - \beta}
	\sqrt{W_1})^2}{\left( \log (\lambda_2 / \lambda_1 ) \right)^2}
\end{equation}

The reader should consult \cite{igeta2018} to understand these equations, as
space constraints disallow a full explanation here. Their chief claim
is that these methods are robust to misspecification of variance.

\section{Software Program}

\subsection{Overview}

We can empirically test these methods by the following procedure.

\begin{itemize}
	\item Randomly generate data that is overdispersed according to some
		true variance function; that has true treatment
		effect $\exp{(\beta_1)}$; and that has sample size equal to that
		recommended by (3) for the desired power $1-\beta$ calculated
		using (2).
	\item Conduct a test of $H_0: \beta_1 = 0$ vs. $H_1: \beta_1 \neq 0$
		using (1). Record whether a type II error occurred.
	\item Conduct many iterations of the previous two steps.
		Calculate the proportion of type II errors:
		$$
		\bar{\beta} = \frac{1}{N}\sum_{t = 1}^N \beta_{(t)}
		$$

		where $\beta_{(t)}$ is an indicator of whether a type II error
		occured for trial $t$ and $N$ is the total number of iterations.
	\item Compare $1-\bar{\beta}$ to the asymptotic power calculated using (2). We
		hope that $1-\bar{\beta} \approx 1 - \beta$.
\end{itemize}

We conduct this procedure for multiple different true variance functions and
working variance functions and effect sizes. If the empirical power $1 - \bar{\beta}$ 
matches the asymptotic power across all specifications, we conclude that the claims of
\cite{igeta2018} are correct and that their methods are robust.

\subsection{Technical Details}

The software program carries out the testing strategy outlined above. Source code is contained in \texttt{final.c}. 
The program requires the GNU Scientific Library (GSL), an
open-source numerical library. It can be obtained from
\url{www.gnu.org/software/gsl}; or, it can installed from most standard Linux
package managers. An example command to achieve the latter is:

\begin{center}
	\texttt{sudo apt-get install gsl-bin libgsl-dev}
\end{center}

Compilation of \texttt{final.c} is best achieved in two steps. First, use the below
command to compile the program but not link it. You may need to change the
argument passed to the \texttt{-I} flag to wherever the \texttt{gsl} header
files live on your computer. 

\begin{center}
	\texttt{gcc -I/usr/include -c final.c}
\end{center}

This command should create an object file \texttt{final.o}. Link this 
object file to relevant libraries with the following command.
You may need to change the argument passed to the \texttt{-L} flag to
wherever \texttt{libgsl} lives on your computer.

\begin{center}
	\texttt{gcc -L/usr/lib final.o -o final -lgsl -lgslcblas -lm}
\end{center}

Once successfully compiled, the program can be executed. It does not require any
arguments. Output is comma-separated text printed to \texttt{stdout}. You likely want to pipe this
output to a text file, as per the following command:

\begin{center}
	\texttt{./final > output.csv}
\end{center}

\section{Simulation Study}

\subsection{Settings}

The rate of event incidence in the control group, $\lambda_1$, is 1.25.
Equivalently, this corresponds to $\beta_0 = 0.22314$ in $\exp{(\beta_0)}$. The
allocation ratio is equal across groups (i.e. $q_1 = q_2 = 0.5$). The follow-up
period $\tau$ is 1 year. Some subjects drop out early. Time to dropout follows
an exponential distribution with parameter value 0.356, such that about 30\% of
subjects drop out early. So in practice the observed follow-up time $T$ is the
miniimum of time to dropout and $\tau$.

The tested values for true relative risk (effect size) $\exp{(\beta_1)}$ are $\exp{(\beta_1)} = \frac{\lambda_2}{\lambda_1}
= 0.4 | 0.6 | 0.8$. The desired power is 90\% across all specifications. 
The power for a particular
true effect size (per above) is calculated using (2). The number of 
observations is computed using (3).

The true variance function is 

\section{Conclusion}

\printbibliography[heading=bibnumbered]

\end{document}
