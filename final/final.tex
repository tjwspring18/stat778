\documentclass{article}

\usepackage{bm}
\usepackage{amsmath}
\usepackage{amssymb}
\usepackage[citestyle=authoryear]{biblatex}
\addbibresource{final.bib}


\author{Tom Wallace}

\title{STAT 778 Final Project: Implementation of Igeta, Takahashi, and Matsui
2018}

\begin{document}

\maketitle

\section{Introduction}
This paper documents an implementation of the technique of \cite{igeta2018}. It
gives some background on their method; presents a fast software program in C
implementing their method; and conducts verification and validation of this
program.

\section{Methods and Code}

\subsection{Background}

Overdispersion refers to a situation in which the variance of a dataset exceeds
that which would be expected under the assumed statistical distribution. 
Overdispersion is common in count data. As a motivating example, consider the
Poisson distribution. A single parameter determines both the 
mean and variance (i.e., for $X \sim \mathrm{Poisson}(\lambda)$, $\mu =
\sigma^2 = \lambda$). As a consequence, if the mean and variance
differ in a dataset, the researcher cannot adjust parametric assumptions for 
without also affecting the other. If not addressed, 
overdispersion results in distorted test statistics and estimated
standard errors. Clinical trials often feature count data: e.g., a trial of an
anti-epsilepsy treatment may use \textit{number of seizures over study period} as the outcome
variable. The serious consequences of statistical error in such settings demands
a rigorous method for dealing with overdispersion. \cite{igeta2018} is a new
entry to the large literature on this topic. In particular, it presents methods
for calculating statistical power and sample size in the presence of
misspecified variance.

\subsection{Methods}

Consider a randomized control trial featuring $n$ subjects. Subjects are
randomly assigned to a treatment or control group. Let $n_i$ be the sample size
of the $i$th group, with $i \in \{1, 2\}$.
Let $q_i = \frac{n_i}{n}$ be the proportion of subjects assigned to
the $i$th group. Let $\pi_i$ be the allocation ratio to the $i$th group, where
$\lim_{n \to \infty} q_i = \pi_i$
Let $X_{ij}$ be an indicator of the treatment assignment for the $j$th subject
($j=1,2\ldots n$) in the $i$th group. $X_{ij}=0$ indicates
assignment to the control group. $X_{ij}=1$ indicates assignment to the
treatment group. Let $Y_{ij}$ be the number of events associated with the subject
during the period $[0, T_{ij}]$, from the point of treatment assignment to the
end of follow-up.

\cite{igeta2018} consider a statistical test for group comparison. The test is
based on a quasi-likelihood using a sandwich-type robust estimator; for further
details, consult their paper. The goal is
to estimate coefficients in a Poisson model:

$$
\lambda_i = \exp{(\beta_0 + \beta_1 X_{ij})}
$$

\cite{igeta2018} propose that the asymptotic power of the test is

\begin{equation}
\mathrm{Pr} \left( Z > z_{1 - \alpha/ 2} \right) 
=
1 - \Phi 
\left(
z_{1 - \alpha / 2}
\sqrt{\frac{W_0}{W_1}}
-
\sqrt{n}
\frac{\beta_1}{\sqrt{W_1}}
\right)
\end{equation}

and the sample size that provides power greater than or equal to 
$1 - \beta$ is

\begin{equation}
n \geq \frac{(z_{1 - \alpha / 2} \sqrt{W_0} + z_{1 - \beta}
\sqrt{W_1})^2}{\left( \log (\lambda_2 / \lambda_1 ) \right)^2}
\end{equation}

where:

\begin{itemize}
\item $z_a$ is the lower $a$ point of the standard normal distrbution
\item $\Phi$ is the CDF of the standard normal distribution
\item When a constant follow-up period $\tau$ 
\end{itemize}


\section{Simulation Study}

\section{Conclusion}

\printbibliography[heading=bibnumbered]

\end{document}
