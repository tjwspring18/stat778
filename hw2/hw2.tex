\documentclass{article}

\usepackage{hyperref}

\author{Tom Wallace}
\title{STAT 778: Homework 2}

\begin{document}
\maketitle

\section{Program Organization}

Code for \textbf{Problem 1} is contained in \texttt{hw2\_1.c}. It relies upon the 
\texttt{math.h} library and must be compiled with the \texttt{-lm} flag. An
example compilation command is:

\begin{center}
	\texttt{gcc hw2\_1.c -o hw2\_1 -lm}
\end{center}

Once compiled, execution of \texttt{hw2\_1} requires one mandatory argument: the name of the
data file to be read. An example command is:

\begin{center}
	\texttt{./hw2\_1 HW2\_2018.dat}
\end{center}

Code for \textbf{Problem 2} is contained in \texttt{hw2\_2.c}. The program
requires the GNU Scientific Library (GSL), an
open-source numerical library. It can be obtained from
\url{www.gnu.org/software/gsl}; or, it can installed from most standard Linux
package managers. An example command of this is:

\begin{center}
	\texttt{sudo apt-get install gsl-bin libgsl-dev}
\end{center}

Compilation of \texttt{hw2\_2.c} is best achieved in two steps. First, use the below
command to compile the program but not link it. You may need to change the
argument passed to the \texttt{-I} flag to wherever the \texttt{gsl} header
files live on your computer. 

\begin{center}
	\texttt{gcc -I/usr/include -c hw2\_2.c}
\end{center}

This command should create an object file \texttt{hw2\_2.o}. Link this 
object file to relevant libraries with the following command.
You may need to change the argument passed to the \texttt{-L} flag to
wherever \texttt{libgsl} lives on your computer.

\begin{center}
	\texttt{gcc -L/usr/lib hw2\_2.o -o hw2\_2 -lgsl -lgslcblas -lm}
\end{center}

Once successfully compiled, the program can be executed. It does not require any
arguments. Output is comma-separated text printed to \texttt{stdout}. You likely want to pipe this
output to a text file, as per the following command:

\begin{center}
	\texttt{./hw2\_2 > output.csv}
\end{center}

\section{Writeup of Problem 2}
\subsection{Introduction}
This document describes a simulation study. The study consisted of generating
statistics based on normal random variables and comparing the accuracy of these
statistics to their theoretical expectation.

\subsection{Methodology}
Consider the normal random variable $X \sim N(-0.5, 2)$. For each simulation
run, $n$ of such variables were randomly generated. Three different $n$ were
used:  $n=50, 100, 200$. For each $n$, 1000 runs were conducted. On each run,
the sample mean and sample variance were computed, as well as their respective standard errors and 95\%
confidence interval. These statistics were averaged over the 1000 runs conducted
for each $n$. The empirical coverage probability also was computed.

\subsection{Results}
Simulation results are presented in Table 1. The point estimates closely match
the true parameter values. The standard errors of the point estimates decrease
as $n$ increases. The empirical coverage probabilities are very near 95\%. 

\begin{table}[h]
	\centering
	\caption{Simulation results, average of 1000 runs}
	\begin{tabular}{|rrrrrrr|}
		\hline
		$n$&Parameter&True Value&Estimate&SE& 95\% CI&CP(\%)\\
		\hline
		50  & $\mu$ & -0.5 & -0.487   & 0.199& (-0.877, -0.097) & 95.6 \\
		    & $\sigma^2$  & 2 & 1.999 & 0.404 & (1.207, 2.790)  & 95.0 \\
		    &&&&&& \\
		100 & $\mu$ & -0.5 & -0.496 & 0.141 & (-0.771, -0.220) & 94.3 \\
		    & $\sigma^2$ & 2 & 1.990 & 0.283 & (1.435, 2.544) &  95.1\\
		    &&&&&& \\
		200 & $\mu$ & -0.5 & -0.498 & 0.100 & (-0.693, -0.302) & 94.9\\ 
		    & $\sigma^2$ & 2 & 1.991 & 0.200 & (1.600, 2.382) & 94.4 \\
		\hline
	\end{tabular}
\end{table}

	\end{document}
