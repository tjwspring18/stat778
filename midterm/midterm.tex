\documentclass{report}

\usepackage{hyperref}

\usepackage{geometry}
%\geometry{margin=1in}

\usepackage{bm}

\usepackage{threeparttable}
\title{STAT 778: Midterm Exam}
\author{Tom Wallace}

\begin{document}

\maketitle

\section*{Preface: Program Organization and Compilation}

Source code is contained in \texttt{midterm.c}. 
The program requires the GNU Scientific Library (GSL), an
open-source numerical library. It can be obtained from
\url{www.gnu.org/software/gsl}; or, it can installed from most standard Linux
package managers. An example command to achieve the latter is:

\begin{center}
	\texttt{sudo apt-get install gsl-bin libgsl-dev}
\end{center}

Compilation of \texttt{midterm.c} is best achieved in two steps. First, use the below
command to compile the program but not link it. You may need to change the
argument passed to the \texttt{-I} flag to wherever the \texttt{gsl} header
files live on your computer. 

\begin{center}
	\texttt{gcc -I/usr/include -c midterm.c}
\end{center}

This command should create an object file \texttt{midterm.o}. Link this 
object file to relevant libraries with the following command.
You may need to change the argument passed to the \texttt{-L} flag to
wherever \texttt{libgsl} lives on your computer.

\begin{center}
	\texttt{gcc -L/usr/lib midterm.o -o midterm -lgsl -lgslcblas -lm}
\end{center}

Once successfully compiled, the program can be executed. It does not require any
arguments. Output is comma-separated text printed to \texttt{stdout}. You likely want to pipe this
output to a text file, as per the following command:

\begin{center}
	\texttt{./midterm > output.csv}
\end{center}

The following R code is useful for analyzing the output data:
\begin{flushleft}
	\texttt{library(plyr)}

	\texttt{df <- read.csv("output.csv", header = T)}
	
	\texttt{analysis <- ddply(df, 
	     c("distribution", "n", "m1", "m2"), 
	     summarize, 
	     emp\_I\_t = mean(I\_t),
	     emp\_I\_w = mean(I\_w),
	     emp\_II\_t = mean(II\_t),
	     emp\_II\_w = mean(II\_w))}
	
	\texttt{df2[df2\$distribution == 1, "distribution"] <- "Normal"}

	\texttt{df2[df2\$distribution == 2, "distribution"] <- "Normal
	(contaminated)"}

	\texttt{df2[df2\$distribution == 3, "distribution"] <- "Exponential"}
\end{flushleft}

\newpage

\section*{Introduction}
This study seeks to compare the performance of the two-sample t-test and the
Wilcoxon rank-sum test. It uses simulation to do so. Data is randomly generated
under different scenarios. For each scenario, the two methods are used to test the 
null hypothesis of no difference of means against the simple alternate hypothesis. 
The goal is to ascertain which method performs better by various criteria.

The remainder of this document is organized into two sections. The
\textbf{Methods} section provides more detail on how the two methods were
implemented and how their performance was compared. The \textbf{Simulation
Study} section presents output data and results.

\section*{Methods}

\subsection*{Tests for Difference of Means}
The study compares the performance of two different tests of means. The first
is \textbf{Welch's t-test}. This test assumes that the two populations
are independent (i.e. unpaired), that they have normal distributions, and that
they may have unequal variances. The test statistic $t$ is calculated as:
\begin{equation}
	t = \frac{\bar{X}_1 - \bar{X}_2}{\sqrt{\frac{s^2_1}{n_1} + \frac{s^2_2}{n_2}}}
\end{equation}

with $\bar{X}_i$, $s^2_i$, and $n_i$ denoting the sample mean, sample variance, and
sample size of group $i$. The degrees of freedom for the $t$ test statistic are
calculated by the Welch-Sattherthwaite equation:

\begin{equation}
	df \approx \frac{\left(\frac{s^2_1}{n_1} + \frac{s^2_2}{n_2}
	\right)^2}{\frac{s^4_1}{n^2_1(n_1 - 1)} + \frac{s^4_2}{n^2_2(n_2 - 1)}}
\end{equation}

The second is the \textbf{Wilcoxon rank-sum test}. This test makes no parametric
assumptions nor any assumptions regarding common variance. Observations from the
two groups are pooled, and then ranked in ascending order. The sum of ranks is
taken for a group (which does not matter). The $u$ statistic is given by:
\begin{equation}
	u_1 = R_1 - \frac{n_1(n_1 + 1)}{2} 
\end{equation}

where $R_1$ is the sum of ranks of group 1, and $n_1$ is the sample size of
group 1. This study uses the normal approximation for groups with $n_i\geq25$:
\begin{equation}
	z = \frac{u - \mu_u}{\sigma_u}
\end{equation}

where $\mu_u = \frac{n_1n_2}{2}$ and $\sigma_u =
\sqrt{\frac{n_1n_2(n_1+n_2+1)}{12}}$

\subsection*{Hypothesis Testing}

The null hypothesis is no difference in group means, with a two-sided alternate
hypothesis:

$$
H_0: \mu_1 = \mu_2
$$
$$
H_1: \mu_1 \neq \mu_2
$$

A significance level of $\alpha = 0.05$ is used.

%\subsection*{Measures of Performance}
%
%The two tests are compared using the following measures, all of which will be
%explored using simulation:
%
%\begin{itemize}
%	\item Type I error: at what rate does the test incorrectly reject the null?
%	\item Power: at what rate does the test correctly reject the null (for various
%		alternate parameter values)?
%	\item Robustness to sample size: how sensitive is the test (in term of type I
%		error and power) to sample size?
%	\item Parametric robustness: how sensitive is the test
%		(in terms of type I error and power) to different distributions?
%	\item Robustness to outliers: how sensitive is the test (in terms of
%		type I error and power) to outliers?
%\end{itemize}

\section*{Simulation Study}
\subsection*{Approach}
The basic approach was to generate simulated data for two
groups according to some specification; apply the t-test
and Wilcoxon rank-sum test to the simulated data; and assess the empirical type I error rate 
and empirical type II error rate of each
test. Different specifications were used, with variation in group
size, distribution (including presence of outliers), and true difference in
means. Each specification was simulated 1000 times. The tested specifications 
included various combinations of:

\begin{itemize}
	\item \textbf{Distribution}
		\begin{itemize}
			\item Normal
			\item Normal, with 2\% chance of outlier
				($\mu_{\mathrm{outlier}} = 100\mu)$
			\item Exponential
		\end{itemize}
	\item \textbf{Group size}
		\begin{itemize}
			\item $n=25$
			\item $n=50$
			\item $n=100$
		\end{itemize}
	\item \textbf{True difference in means}
		\begin{itemize}
			\item None ($\mu_2 = \mu_1$)
			\item $\mu_2 = 1.05\mu_1$
			\item $\mu_2 = 1.5\mu_1$
			\item $\mu_2 = 2\mu_1$
		\end{itemize}
\end{itemize}

\subsection*{Results}

Quantitative results are presented in Table 1. 

The t-test and Wilcoxon rank-sum
test perform about the same on normally-distributed data. The power of the tests
is low for small true difference in means and improves at a similar rate as the
true difference in means becomes greater.

The Wilcoxon rank-sum test greatly outperforms the t-test on normal
data with outliers. The power of the t-test is low and scarcely improves with
increased true difference in means; in contrast, the Wilcoxon rank-sum test is
much more powerful. This simulation finding makes theoretical sense: the
t-test is based on means (which are sensitive to outliers) whereas the
Wilcoxon rank-sum test is based on medians (which are much less
sensitive to outliers). This finding is true for all tested sample sizes.

The t-test seems to be more powerful than the Wilcoxon rank-sum test on exponential data,
particularly as sample size increases. This is surprising as normality is an
assumption of the t-test. The Wilcoxon test performs about the same on the
exponential data as it does on other distributions; this is theoretically
expected as the test makes no parametric assumptions.

\begin{table}[h!]
	\centering
	\caption{Simulation Results}
	\vspace{1em}
	\begin{threeparttable}
		\begin{tabular}{|l r r r r r r r|}
			\hline
			& & & & & & & \\
			\textbf{Distribution} & 
			\textbf{n} & 
			$\bm{\mu}_1$ &
			$\bm{\mu}_2$ &
			$\bar{\bm{\alpha}_t}$ & 
			$\bar{\bm{\alpha}}_u$ & 
			$\bar{\bm{\beta}}_t$ &
			$\bar{\bm{\beta}}_u$ \\

			& & & & & & & \\

			Normal &  25 & 1.00 & 1.00 & 0.04 & 0.05 & - & - \\ 
			&  & & 1.05 & - & - & 0.95 & 0.96 \\ 
			&  & & 1.50 & - & - & 0.58 & 0.59 \\ 
			&  & & 2.00 & - & - & 0.07 & 0.07 \\ 
			&  50 & 1.00 & 1.00 & 0.05 & 0.05 & - & - \\ 
			&  & & 1.05 & - & - & 0.94 & 0.94 \\ 
			&  & & 1.50 & - & - & 0.31 & 0.34 \\ 
			&  & & 2.00 & - & - & 0.00 & 0.00 \\ 
			& 100 & 1.00 & 1.00 & 0.05 & 0.05 & - & - \\ 
			& & & 1.05 & - & - & 0.93 & 0.94 \\ 
			& & & 1.50 & - & - & 0.06 & 0.07 \\ 
			& & & 2.00 & - & - & 0.00 & 0.00 \\ 
			Normal (contaminated) &  25 & 1.00 & 1.00 & 0.02 & 0.04
			& - & - \\ 
			&   & & 1.05 & - & - & 0.98 & 0.95 \\ 
			&   & & 1.50 & - & - & 0.84 & 0.62 \\ 
			&   & & 2.00 & - & - & 0.67 & 0.11 \\ 
			&  50 & 1.00 & 1.00 & 0.01 & 0.05 & - & - \\ 
			&   & & 1.05 & - & - & 0.98 & 0.94 \\ 
			&   & & 1.50 & - & - & 0.91 & 0.34 \\ 
			&   & & 2.00 & - & - & 0.85 & 0.00 \\ 
			& 100 & 1.00 & 1.00 & 0.04 & 0.06 & - & - \\ 
			&  & & 1.05 & - & - & 0.96 & 0.94 \\ 
			&  & & 1.50 & - & - & 0.94 & 0.09 \\ 
			&  & & 2.00 & - & - & 0.91 & 0.00 \\ 
			Exponential &  25 & 1.00 & 1.00 & 0.04 & 0.05 & - & - \\ 
			&   & & 1.05 & - & - & 0.96 & 0.95 \\ 
			&   & & 1.50 & - & - & 0.73 & 0.76 \\ 
			&   & & 2.00 & - & - & 0.39 & 0.48 \\ 
			&  50 & 1.00 & 1.00 & 0.06 & 0.06 & - & - \\ 
			&   & & 1.05 & - & - & 0.94 & 0.95 \\ 
			&   & & 1.50 & - & - & 0.50 & 0.59 \\ 
			&   & & 2.00 & - & - & 0.09 & 0.15 \\ 
			& 100 & 1.00 & 1.00 & 0.04 & 0.04 & - & - \\ 
			&  & & 1.05 & - & - & 0.94 & 0.94 \\ 
			&  & & 1.50 & - & - & 0.19 & 0.33 \\ 
			&  & & 2.00 & - & - & 0.00 & 0.01 \\ 

			\hline
		\end{tabular}
		\begin{tablenotes}
		\item Each specification simulated 1000 times
		\item $\sigma^2 = 1$ for all normal and contaminated normal
			distributions
		\item $\bar{\alpha}_t$ = empirical Type I error probability, t-test
		\item $\bar{\alpha}_u$ = empirical Type I error probability,
			Wilcoxon rank-sum test
		\item $\bar{\beta}_t$ = empirical Type II error probability, t-test
		\item $\bar{\beta}_u$ = empirical Type II error probability, Wilcoxon rank-sum test
		\end{tablenotes}
	\end{threeparttable}
\end{table}

\clearpage
\section*{Comments Regarding Course}

Overall, I have a positive impression of the class thus far. My C programming
skills have gotten much sharper. I particularly appreciate that the instructor
focuses on more on students learning useful skills than on grades. The purpose of
graduate school is to prepare students to do professional research, and so 
I feel that grades are relatively superfluous. They can be a useful
barometer of whether students are learning the skills needed for research, but the
relationship is not that strong. In some other courses, the pressure 
to do well on frequent, difficult graded assignments actually hurt my
progress as a statistician. I got excellent grades, but did so by focusing more on 
learning to crank out correct answers than on truly understanding the material.
I really appreciate the different approach of this class. I put in just as much 
work, but am free to do so in a way that aids my long-term progress. 

I hope that in the second half of the course we spend more time on algorithms.
Techniques such as jackknife, bootstrap, EM, and the like are fundamental to
modern statistics. Most of our work in the first half of the class has been on
learning C, not learning statistical algorithms---we have been implemented very
basic procedures such as Kaplan-Meier. I look forward to learning more advanced
algorithms.

\end{document}

