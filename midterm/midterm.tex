\documentclass{report}

\usepackage{hyperref}

\title{STAT 778: Midterm Exam}
\author{Tom Wallace}

\begin{document}

\maketitle

\section*{Preface: Program Organization and Compilation}

Source code is contained in \texttt{midterm.c}. 
The program requires the GNU Scientific Library (GSL), an
open-source numerical library. It can be obtained from
\url{www.gnu.org/software/gsl}; or, it can installed from most standard Linux
package managers. An example command to achieve the latter is:

\begin{center}
	\texttt{sudo apt-get install gsl-bin libgsl-dev}
\end{center}

Compilation of \texttt{midterm.c} is best achieved in two steps. First, use the below
command to compile the program but not link it. You may need to change the
argument passed to the \texttt{-I} flag to wherever the \texttt{gsl} header
files live on your computer. 

\begin{center}
	\texttt{gcc -I/usr/include -c midterm.c}
\end{center}

This command should create an object file \texttt{midterm.o}. Link this 
object file to relevant libraries with the following command.
You may need to change the argument passed to the \texttt{-L} flag to
wherever \texttt{libgsl} lives on your computer.

\begin{center}
	\texttt{gcc -L/usr/lib midterm.o -o midterm -lgsl -lgslcblas -lm}
\end{center}

Once successfully compiled, the program can be executed. It does not require any
arguments. Output is comma-separated text printed to \texttt{stdout}. You likely want to pipe this
output to a text file, as per the following command:

\begin{center}
	\texttt{./midterm > output.csv}
\end{center}

\newpage

\section*{Introduction}
This study seeks to compare the performance of the two-sample t-test and the
Wilcoxon rank-sum test. It uses simulation to do so. Data is randomly generated
under different scenarios. For each scenario, the two methods are used to test the 
null hypothesis of no difference of means against the simple alternate hypothesis. 
The goal is to ascertain which method performs better by various criteria.

The remainder of this document is organized into two sections. The
\textbf{Methods} section provides more detail on how the two methods were
implemented and how their performance was compared. The \textbf{Simulation
Study} section presents output data and results.

\section*{Methods}

\subsection*{Tests for Difference of Means}
The study compares the performance of two different tests of means. The first
is \textbf{Welch's t-test}. This test assumes that the two populations
are independent (i.e. unpaired), that they have normal distributions, and that
they may have unequal variances. Although Welch's t-test is
possible to perform with different population sizes, this study only examined
the case of equal population sizes. The test statistic $t$ is calculated as:
\begin{equation}
t = \frac{\bar{X}_1 - \bar{X}_2}{\sqrt{\frac{s^2_1}{n_1} + \frac{s^2_2}{n_2}}}
\end{equation}

with $\bar{X}_i$, $s^2_i$, and $n_i$ denoting the sample mean, sample variance, and
sample size of group $i$. The degrees of freedom for the $t$ test statistic are
calculated by the Welch-Sattherthwaite equation:

\begin{equation}
df \approx \frac{\left(\frac{s^2_1}{n_1} + \frac{s^2_2}{n_2}
\right)^2}{\frac{s^4_1}{n^2_1(n_1 - 1)} + \frac{s^4_2}{n^2_2(n_2 - 1)}}
\end{equation}

The second is the \textbf{Wilcoxon rank-sum test} (sometimes also called
the Mann-Whitney U test). This test---or at least this paper's implementation of
it---assumes independent (i.e. unpaired) samples, but makes no parametric
assumptions nor any assumptions regarding common variance. Observations from the
two groups are pooled, and then sorted and ranked in ascending order (i.e. the smallest
observation is 1, the second-smallest is 2, and so on). 
\begin{itemize}
\item Type I error
\item Power 
\item Robustness to sample size
\item Robustness to false parametric assumptions
\item Robustness to outliers
\end{itemize}
\end{document}

