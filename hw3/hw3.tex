\documentclass{article}

\usepackage{hyperref}

\usepackage{geometry}
\geometry{margin=1in}

\usepackage{amsmath}
\usepackage{amssymb}
\usepackage{bm}

\author{Tom Wallace}

\title{STAT 778 HW 3}

\begin{document}

\maketitle

\section*{Program Organization and Compilation}

Source code is contained in \texttt{hw3.c}. 
The program requires the GNU Scientific Library (GSL), an
open-source numerical library. It can be obtained from
\url{www.gnu.org/software/gsl}; or, it can installed from most standard Linux
package managers. An example command to achieve the latter is:

\begin{center}
	\texttt{sudo apt-get install gsl-bin libgsl-dev}
\end{center}

Compilation of \texttt{hw3.c} is best achieved in two steps. First, use the below
command to compile the program but not link it. You may need to change the
argument passed to the \texttt{-I} flag to wherever the \texttt{gsl} header
files live on your computer. 

\begin{center}
	\texttt{gcc -I/usr/include -c hw3.c}
\end{center}

This command should create an object file \texttt{hw3.o}. Link this 
object file to relevant libraries with the following command.
You may need to change the argument passed to the \texttt{-L} flag to
wherever \texttt{libgsl} lives on your computer.

\begin{center}
	\texttt{gcc -L/usr/lib hw3.o -o hw3 -lgsl -lgslcblas -lm}
\end{center}

Once successfully compiled, the program can be executed. It requires one
argument, the name of the input file. An example command is:

\begin{center}
	\texttt{./hw3 HW2\_2018.dat}
\end{center}

\section*{Technical Notes}

This program finds maximum partial likelihood estimates (MPLE) for coefficients
$\beta_j\ldots\beta_k=\bm{\beta}$ in the Cox proportional hazards model. This
section provides the details of the computation of these estimates.

The log-likelihood function for the Cox proportional hazards model is:

\begin{equation}
\log L(\bm{\beta}) = \sum_{i=1}^n \delta_i \left( \bm{\beta'x}_i - \log \sum_{l
\in R(t_i)} \exp{(\bm{\beta' x}_l)}\right)
\end{equation}

with:
\begin{tabular}{l c l}
$t_i...t_n$ & = & unique observation times \\
$\delta_i$ & = & indicator function, which returns 0 if observation $t_i$ is
right-censored \\
$\bm{x}_i$ & = & vector of covariates associated with individual observed at
$t_i$ \\
$\bm{\beta}$ & = & vector of coefficients associated with covariates \\
$R(t_i)$ & = & risk set at $t_i$
\end{tabular}

\bigskip

The first order partial derivatives are:

\begin{equation}
\frac{\partial \log L}{\partial \beta_j} = 
\sum_{i=1}^n x_j - \frac{\sum_{l \in R(t_i)} x_{lj} \exp(\bm{\beta'x}_l)}{ \sum_{l
\in R(t_i)} \exp(\bm{\beta'x}_l)}
\end{equation}

It can be shown that these functions are convex, and so the their roots---i.e.,
the value of $\beta_k$ for which the above function equals 0---is the MPLE
estimator $\hat{\beta}_k$. The Newton-Raphson algorithm is used to numerically
estimate the roots (there is no closed-form solution). This algorithm also
requires the derivatives of the function to be solved. Thus, we also need the
second-order partial derivatives of the log likelihood function.

\begin{equation}
\frac{\partial^2 \log L}{\beta_j \beta_k} = 
\sum_{i=1}^n 
\frac{1}{\sum_{l \in R(t_i)} \exp(\bm{\beta'x}_l)}
\left(
\sum_l x_{lj}x_{lk} \exp(\bm{\beta'x}_l)- 
\frac{
\left(\sum_l x_{lj} \exp(\bm{\beta'x}_l)\right)
\left(\sum_l x_{lk} \exp(\bm{\beta'x}_l)\right)
}
{\sum_l \exp(\bm{\beta'x}_l)}
\right)
\end{equation}

The program uses these expressions to find the first- and second-order partial
derivatives of the log likelihood function and supplies them to the
Newton-Raphson algorithm.

\section*{Verification and Validation}

\end{document}
