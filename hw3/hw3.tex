\documentclass{article}

\usepackage{hyperref}

\usepackage{geometry}
\geometry{margin=1in}

\usepackage{amsmath}
\usepackage{amssymb}
\usepackage{bm}

\author{Tom Wallace}

\title{STAT 778 HW 3}

\begin{document}

\maketitle

\section*{Program Organization and Compilation}

Source code is contained in \texttt{hw3.c}. 
The program requires the GNU Scientific Library (GSL), an
open-source numerical library. It can be obtained from
\url{www.gnu.org/software/gsl}; or, it can installed from most standard Linux
package managers. An example command to achieve the latter is:

\begin{center}
	\texttt{sudo apt-get install gsl-bin libgsl-dev}
\end{center}

Compilation of \texttt{hw3.c} is best achieved in two steps. First, use the below
command to compile the program but not link it. You may need to change the
argument passed to the \texttt{-I} flag to wherever the \texttt{gsl} header
files live on your computer. 

\begin{center}
	\texttt{gcc -I/usr/include -c hw3.c}
\end{center}

This command should create an object file \texttt{hw3.o}. Link this 
object file to relevant libraries with the following command.
You may need to change the argument passed to the \texttt{-L} flag to
wherever \texttt{libgsl} lives on your computer.

\begin{center}
	\texttt{gcc -L/usr/lib hw3.o -o hw3 -lgsl -lgslcblas -lm}
\end{center}

Executing the program requires one
argument, the name of the input file. An example command is:

\begin{center}
	\texttt{./hw3 HW2\_2018.dat}
\end{center}

\section*{Technical Notes}

This program finds maximum partial likelihood estimates (MPLE) for coefficients
$\bm{\beta}$ in the Cox proportional hazards model. This
section provides mathematical background on the computation of these estimates.

The log-likelihood function for the Cox proportional hazards model is:

\begin{equation}
\log L(\bm{\beta}) = \sum_{i=1}^n \delta_i \left( \bm{\beta'x}_i - \log \sum_{l
\in R(t_i)} \exp{(\bm{\beta' x}_l)}\right)
\end{equation}

with:
\begin{tabular}{l c l}
$t_i...t_n$ & = & unique observation times \\
$\delta_i$ & = & indicator function, which returns 0 if observation $t_i$ is
right-censored \\
$\bm{x}_i$ & = & vector of covariates associated with individual observed at
$t_i$ \\
$\bm{\beta}$ & = & vector of coefficients associated with covariates \\
$R(t_i)$ & = & risk set at $t_i$
\end{tabular}

\bigskip

The first order partial derivatives are:

\begin{equation}
\frac{\partial \log L}{\partial \beta_j} = 
	\sum_{i=1}^n 
	\delta_i
	\left( 
	x_{ij} - \frac{\sum_{l \in R(t_i)} x_{lj} \exp(\bm{\beta'x}_l)}
	{ \sum_{l \in R(t_i)} \exp(\bm{\beta'x}_l)}
	\right)
\end{equation}

It can be shown that these functions are concave, and so the their roots---i.e.,
the values of $\bm{\beta}$ for which the above functions equal $\bm{0}$---are the MPLE
estimators $\hat{\bm{\beta}}$. The Newton-Raphson algorithm is used to numerically
estimate the roots (there is no closed-form solution). This algorithm also
requires the derivatives of the function to be solved. Thus, we also need the
second-order partial derivatives of the log likelihood function.

\begin{equation}
\frac{\partial^2 \log L}{\partial \beta_j \partial \beta_k} = 
-
\sum_{i=1}^n \
\delta_i
	\left( 
	\frac{\sum_{l \in R(t_i)} x_{lj}x_{lk} \exp(\bm{\beta'x}_l)}
	{\sum_{l \in R(t_i)} \exp(\bm{\beta'x}_l)}
- 
\frac{
	\left(\sum_{l \in R(t_i)} x_{lj} \exp(\bm{\beta'x}_l)\right)
	\left(\sum_{l \in R(t_i)} x_{lk} \exp(\bm{\beta'x}_l)\right)
}
	{
		\left(
		\sum_{l \in R(t_i)} \exp(\bm{\beta'x}_l)
		\right)^2
		}
	\right)
\end{equation}

Once we have the Jacobian $\bm{J}$ and Hessian $\bm{H}$, the multivariate
Newton-Raphson algorithm is:

\begin{equation}
	\bm{\beta}_{n+1} = \bm{\beta}_n -
	\bm{H}^{-1}(\bm{\beta}_n)\bm{J}(\bm{\beta}_n)
\end{equation}

In our case of two covariates:
\begin{equation}
	\begin{bmatrix}
		\beta_{1, n+1} \\
		\beta_{2, n+1} \\
	\end{bmatrix}
	=
	\begin{bmatrix}
		\beta_{1, n} \\
		\beta_{2, n} \\
	\end{bmatrix}
	-
	\begin{bmatrix}
		\frac{\partial^2 \log L}{\partial \beta_1^2}\left( \beta_{1,
		n}, \beta_{2, n} \right) &

		\frac{\partial^2 \log L}{\partial \beta_1 \partial \beta_2}\left( \beta_{1,
		n}, \beta_{2, n} \right) 
		
		\\

		\frac{\partial^2 \log L}{\partial \beta_2 \beta_1}\left( \beta_{1,
		n}, \beta_{2, n} \right) &

		\frac{\partial^2 \log L}{\partial \beta_2^2}\left( \beta_{1,
		n}, \beta_{2, n} \right) 
	\end{bmatrix}^{-1}
	\begin{bmatrix}
		\frac{\partial \log L}{\partial \beta_1}\left( \beta_{1, n},
		\beta_{2, n}\right) \\
		\frac{\partial \log L}{\partial \beta_2} \left( \beta_{1, n},
		\beta_{2, n} \right)\\
	\end{bmatrix}
\end{equation}

The inverse Hessian $\bm{H}^{-1}$ is obtained using LU decomposition. The algorithm 
iterates until an arbitrary quality of approximation is obtained: $10e^{-6}$ was
used.

\begin{equation}
	\bm{\beta}_{n} - \bm{\beta}_{n-1} \leq \epsilon
\end{equation}

\section*{Verification and Validation}
This program's output was compared to that obtained using
the \texttt{coxph} function from the \texttt{survival} package in R. As depicted
in Table 1, the two are essentially identical.

\begin{table}[h!]
	\caption{Verification and validation}
	\centering
	\begin{tabular}{|l r r r|}
		\hline
		& & & \\
		& $\hat{\beta}_1$ & $\hat{\beta}_2$  & Log likelihood\\
		& & & \\
		\textbf{R \texttt{survival} package} & 0.0405 & 0.0607&           \\
		& & & \\
		\textbf{This C program} & 0.0405  &  0.0605&       \\
		& & & \\
		\hline
	\end{tabular}
\end{table}

\end{document}
